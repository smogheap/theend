\documentclass{screenplay}
\title{The End}
\author{Micah Gorrell and Owen Swerkstrom}
\realauthor{Owen Swerkstrom}
\address{3451 Peregrine Road\\
Eagle Mountain, UT 84005\\
801-789-2915}
\begin{document}
\coverpage
\fadein
\intslug[]{Spaceship}
The floor curves upwards in a complete circle; the craft's spin is used for artificial gravity.  Inside and out, the ship is a patchwork.  Panels are out of place, wires and tubes and racks of parts and half-finished projects seem to be everywhere.  The front wall slopes out in a cone shape, and the back wall is flat.  The ship measures roughly 15' front to back, and 20' across.

Like much of the interior, the back wall is covered by various panels and wiring.  It also has a large docking port and a stasis pod.  The back wall's stripped layout is more similar to a floor than a side wall.

Lights flicker, and a monitor can be seen displaying an error message.
\begin{dialogue}{}
WARNING: Electrical failure
Panel 0: 0.0 Volts
\end{dialogue}
The stasis pod's magnetic lock disengages.  Its door swings open slowly and heavily.  MAN, unconscious and hooked up to various tubes, falls out onto the floor, along with several gallons of fluid and some medical apparatus.
\begin{dialogue}[groaning, mumbling]{Man}
Whe-?  No mor-
\end{dialogue}
MAN makes a weak attempt to lift his head from the floor.  MAN throws up violently and passes out, landing in the mess of crushed medical gear, stasis fluid, and vomit.
\fadeout
\fadein
An unknown amount of time has passed and MAN again tries to get up.  The only lights on are an emergency bulb and a computer monitor.

MAN grabs some towels from near the stasis pod.  Still in a haze, he starts to clean himself off.  The towel catches an IV which tears out of his arm and bleeds.  MAN removes the remaining needles from his arms, and notices a power cable running through the puddle of fluid and vomit.  He picks it up to move it away.  MAN wipes off the cable, and does some cursory cleaning up.  MAN keeps one towel wrapped around himself and drags the power cable away from the mess.

MAN, still disoriented, makes his way to a seat by the sole active monitor.
\begin{dialogue}[very gruffly]{Man}
Hey doc.  What's up?
\end{dialogue}
MAN waits as if expecting a response.  When he gets none, he taps impatiently on the monitor.
\begin{dialogue}{Man}
Doc?  You alive?
\end{dialogue}
\begin{dialogue}[on-screen]{Doc}
DOC system active.
Powersave mode.
\end{dialogue}
\begin{dialogue}{Man}
What's with the silent treatment, Doc?
\end{dialogue}
\begin{dialogue}[on-screen]{Doc}
Testing audio systems... [PASS]
Speakers may be off, disconnected, or damaged.
\end{dialogue}
\begin{dialogue}{Man}
Ssspeakers.  'Kay...
\end{dialogue}
MAN looks around, still drying off, trying to remember what is where.  He fumbles around behind a panel and finds a speaker.  He reconnects a wire that has come loose, then tosses the speaker back behind the panel.
\begin{dialogue}{Man}
Better?
\end{dialogue}
A short tone is heard.
\begin{dialogue}[aloud, synthetic voice]{Doc}
Speaker output has been restored.
\end{dialogue}
\begin{dialogue}{Man}
Good.  Where are we?
\end{dialogue}
\begin{dialogue}{Doc}
Unknown.  Minimal power is available.  External cameras are inactive.
\end{dialogue}
\begin{dialogue}{Man}
Lovely.  What do I need to fix?
\end{dialogue}
\begin{dialogue}{Doc}
Solar panel 0 may be disconnected or damaged.
\end{dialogue}
\begin{dialogue}[sobering up]{Man}
So I've got to climb out there.  Let me guess, I've got to retract the sails by hand too?
\end{dialogue}
\begin{dialogue}{Doc}
Insufficient power remains to fully retract the sails.  Manual retraction will be required.
\end{dialogue}
\begin{dialogue}{Man}
Next time just say ``yes''.
\end{dialogue}
\extslug[]{Spaceship}
The ship is spinning, and gives the impression of moving through space like a screw.  From MAN's perspective grounded to the ship, stars in the background spin by dizzyingly fast.

MAN is exiting through a door on the back side of the ship's nosecone.  He is wearing a space suit that appears to have been pieced together from several sources.  The main suit is orange, but one arm is blue, and the helmet is red.  One of the legs is bunched up as if it is longer than the other.

MAN closes the door and begins pulling himself along the main support pole that runs the length of the sail.  He reaches the first set of sails, inserts a hand crank, and slowly starts turning it.

Eventually all the sails are retracted and MAN is able to reach the end of the pole.  As he advances along, minor damage to the sails and evidence of the ship's hand-made nature are visible.  Exposed wires run the length of the pole and are held in place with duct tape and zip ties.  Some welds holding the sails' rods are sloppy.

At the end of the pole is a flat disc supporting a solar panel 3' across and an antenna.  The disc is dented and scraped as if by a bullet.  MAN fiddles with the panel's wires for a moment until a small LED on the back of the disc lights up.
\begin{dialogue}[over radio to MAN's helmet]{Doc}
Solar panel 0 is active.  External cameras are on-line.
\end{dialogue}
\begin{dialogue}{Man}
All right.  Now figure out where we are.
\end{dialogue}
MAN begins climbing back towards the cone.
\begin{dialogue}[over radio]{Doc}
Position has been determined with 7 per cent error.  Location is 1536.34 light years from Earth, approaching the Deneb system.
\end{dialogue}
\begin{dialogue}[laughing loudly]{Man}
Right!  Of course, Deneb, how'd I miss that?  Come on Doc.  What's the next match, and just out of curiosity, what was the 7 per cent error?
\end{dialogue}
\begin{dialogue}[over radio]{Doc}
No other positions can be calculated.  5 per cent error is due to background noise.  2 per cent error is due to the absence of light from Sol.
\end{dialogue}
\begin{dialogue}[confused, then very scared]{Man}
What?
\end{dialogue}
CUT TO BLACK

OPENING SEQUENCE

\fadein
\intslug[]{Spaceship}
MAN is climbing down a ladder from the airlock, and taking off his helmet.  As he leaves the ladder, MAN trips over the power cable he'd moved earlier.
\begin{dialogue}{Man}
What do you mean no light from Sol?  As in the sun?  Doc!  What's going on?
\end{dialogue}
\begin{dialogue}[after a pause]{Man}
Doc?
\end{dialogue}
MAN notices he's unplugged the ship's main computer by tripping over an extension cord which was connected to another extension cord.
\begin{dialogue}[under his breath]{Man}
oh for Fry's sake...
\end{dialogue}
MAN ties the cables in a knot then plugs them together again, moving them away from the ladder.
\begin{dialogue}{Man}
Doc?
\end{dialogue}
\begin{dialogue}{Doc}
Main systems on-line.  Initialization and hardware checks in progress.
\end{dialogue}
\begin{dialogue}{Man}
What were you saying about there being no light from the sun?
\end{dialogue}
\begin{dialogue}[after a pause]{Doc}
Unknown.  Permanent storage initialization in progress.  Radiation level is elevated.  Stasis is recommended until a benign radiation level is reached.  (pause)  Permanent storage is off-line.  Stasis hardware -
\end{dialogue}
MAN checks around behind the panel where he previously threw the speaker.
\begin{dialogue}[interrupting]{Man}
Wait, what do you mean permanent storage is off-line?  Is \underline{every} cable getting unplugging today?
\end{dialogue}
\begin{dialogue}{Doc}
Store 0 is attached.  All data requests fail.  Warranty and expected longevity of the device have lapsed.  (pause)  Absence of Sol confirmed.  Location is 1536.34 light years from Earth, approaching the Deneb system.  Stasis hardware check failed.  Warranty and expected -
\end{dialogue}
\begin{dialogue}[slowly, overwhelmed]{Man}
Doc, what is today?
\end{dialogue}
\begin{dialogue}{Doc}
The current UTC time is 18559, September 19th, 3:34.
\end{dialogue}
\begin{dialogue}{Man}
Tell me your clock is also out of warranty.
\end{dialogue}
\begin{dialogue}{Doc}
Clock value is consistent with astral positions, with 7 per cent -
\end{dialogue}
\begin{dialogue}[interrupting]{Man}
I'm going to die, aren't I?
\end{dialogue}
\begin{dialogue}{Doc}
No immediate threat is detected.  Prolonged exposure to current radiation level may cause illness and death.  Stasis is recommended until a benign radiation level is reached.  Stasis hardware check -
\end{dialogue}
MAN looks at the torn tubes and other broken stasis equipment.
\begin{dialogue}[interrupting]{Man}
Next time just say ``yes''.
\end{dialogue}
\extslug{Spaceship}
The ship's solar sails automatically extend.  Each rolls out smoothly in turn, in contrast to MAN's laborious crank-driven retraction.  From a distance, the ship seems to be of reasonably professional, if spartan, design, engineered with good precision.
\intslug{Spaceship}
MAN talks to himself as he walks in complete circles around the ship's interior.  As he walks, he picks up and fidgets with various objects.  The ship's artificial gravity is seen in full, keeping objects he drops or sets down fixed to the ship's floor as it curves behind, up, around, and back to the front of MAN as he moves.
\begin{dialogue}[to himself]{Man}
I'm 16,000 years away from Earth, which is probably gone, along with the sun.  Waking up choking on my own puke, that I can handle.  Waking up to the sun being gone, that makes me wonder whether my bad luck is a sort of super power I could use to fight crime.  Sun's gone!  Ha ha!  Why?  Who knows.  Any records I might have of what happened are on a dead store, or might have been in active memory, until I unplugged it.  Evil-doers beware, my bad luck will foil your plots faster than...  (pause) For all I know, I'm the last human being in the universe.  The only thing that kept me alive was a stasis pod, which I've broken.  Radiation will start to kill me before too long.  I can stave that off within the stasis pod's shielding, but even then, I'll never live long enough to get anywhere worth going to.  If in fact there is anywhere worth going to.  Maybe I'll get lucky and be rescued by a floating island inhabited by xenophilic alien babes.  Or maybe I'll sneeze and my ship will pop, sending my corpse out to get chopped up into deli-thin slices by a planet's rings on its way down to create a lovely atmo-fire display sparking a holy war and killing off a thousand alien species.
\end{dialogue}
\intercut
\intslug{Spaceship}
MAN tinkers with things around the ship.  MAN cleans up and disinfects the mess he made when leaving stasis.  MAN tries to use stacks of memory tabs and data cards in the ship's computer, none of which read.  MAN observes very closely some of the plankton plates that keep the ship's atmosphere balanced.  Some are dark and presumably dead, some have bulged and cracked their covering and are leaking out.  MAN tapes down cables along corners.  MAN organizes the stasis pod's equipment into broken and salvageable piles, parts laid out vaguely where he found them.  MAN sits on a chair in something like a fetal position, staring, numb with worry.
\extslug{Spaceship}
Up close, the ship's homemade and cobbled nature is obvious.  Everything attached to the nosecone (six plasma engines, the central sail spine, several magnetic utility compartments) looks particularly after-market.  The central sail core and surrounding structure are structurally sound, but have developed cracks and scars from age and debris over thousands of years.  The ship was never meant to operate over such great distance and time.
\intslug{Spaceship}
MAN is sitting back comfortably, examining one of the broken pieces from the stasis pod.
\begin{dialogue}{Man}
Doc, how much do your factory settings know about my stasis pod?  Can I fix it?
\end{dialogue}
\begin{dialogue}{Doc}
No source schematics of any stasis hardware are on file.  4 of 4 bio-cycle devices are attached and on-line.  1 of 16 sensors are active.  Remaining sensors may be disconnected or damaged.  2 of 12 actuators are active.  Remaining actuators be disconnected or damaged.
\end{dialogue}
MAN looks over the pod and his collections of parts, trying to make sense of what DOC is saying.
\begin{dialogue}{Man}
How much of this stuff is vital?  If I can't fix or replace some of these, would I still live?
\end{dialogue}
\begin{dialogue}{Doc}
Sensors 4, 5, 6 and 7 are redundant of sensors 0, 1, 2 and 3 respectively.  Sensor 15 is reserved.  Other sensors are unique.  Medical necessity of all sensors should be assumed.  Actuators 8, 9, 10 and 11 provide fail-over for the four throttles of actuator 6.  Other actuators are unique.  Medical necessity of all actuators should be assumed.
\end{dialogue}
\begin{dialogue}[after contemplation]{Man}
I'm gonna die, aren't I?
\end{dialogue}
\begin{dialogue}{Doc}
Yes.
\end{dialogue}
CUT TO BLACK

\extslug{Spaceship}
\begin{dialogue}{Man (V.O.)}
Captain's log, stardate One-eighty-five fifty-something, point... September.  In a shocking reversal of fortune, I found the stasis pod's manual.
\end{dialogue}
\intslug{Spaceship}
\begin{dialogue}[cont.]{Man}
To balance out this bit of cosmic good luck, it's actually what appears to be the manual of a similar model, and has a target audience of highly-trained military medics and engineers.  I majored in software.  Which, even if I could write some code that would help, I've got no working stores to save it to.  ...Speaking of which, Doc, are you recording this in any way?
\end{dialogue}
\begin{dialogue}{Doc}
Ten minutes of audio spool are buffered, awaiting store availability.  Rollover data is freed.  Parsed speech spool is buffered, awaiting store availability.  Buffer is four per cent full.
\end{dialogue}
\begin{dialogue}{Man}
You might be waiting a while for that store availability, unless you know of a Byte Botique around here.  Any Denebian strip malls in the area?
\end{dialogue}
\begin{dialogue}{Doc}
Unknown.  Clarification is -
\end{dialogue}
\begin{dialogue}[frustrated]{Man}
That was a joke, Doc!  You knew that!  I had you all set up, you knew more about me than my public key!  And whatever you learn now, you'll forget if and when we lose power again...  Which I shouldn't worry about since I'm gonna die alone in deep space soon anyway!  The end!  Cut!  Print!  ...Wait a sec...  What's the expected lifespan of a bytepage?
\end{dialogue}
MAN looks through some compartments, finds a thin bytepage, and plugs it into the console.
\begin{dialogue}{Doc}
Unknown.
\end{dialogue}
\begin{dialogue}{Man}
Run a full write check on this, eh?  There's nothing on it I care about.
\end{dialogue}
\begin{dialogue}{Doc}
Consistency check succeeded with 661 bad segments.
\end{dialogue}
\begin{dialogue}[cautiosly optimistic]{Man}
I think it had had that when I got it!  Can you use this as a store?  I know it's small, but... just the important stuff, you know?
\end{dialogue}
\begin{dialogue}{Doc}
Bytepage 0 does not support store requests.
\end{dialogue}
\begin{dialogue}{Man}
You can't bridge them?  Just wrap the reads and writes, we don't need the whole store spec.
\end{dialogue}
\begin{dialogue}{Doc}
No means of autonomous software construction are permitted with the current visor.
\end{dialogue}
\begin{dialogue}{Man}
Dammit, none of the...  Wait.  Can you run unsigned code?
\end{dialogue}
\begin{dialogue}{Doc}
No.
\begin{dialogue}{Man}
Oh, for...  Oh!  You're burned with my public key right?  If I built and signed a package, you could run it?
\end{dialogue}
\begin{dialogue}{Doc}
Yes.
\end{dialogue}
\begin{dialogue}{Man}

\end{dialogue}
\begin{dialogue}{Doc}
\end{dialogue}
\begin{dialogue}{Man}
\end{dialogue}
\begin{dialogue}{Doc}
\end{dialogue}
\begin{dialogue}{Man}
\end{dialogue}
\begin{dialogue}{Doc}
\end{dialogue}
\begin{dialogue}{Man}
...But it could.  You've still got a compiler, right?
\end{dialogue}
\begin{dialogue}{Doc}
Yes.
\end{dialogue}
\begin{dialogue}{Man}
Pull up the docs for the store and bytepage plugs.
\end{dialogue}
\begin{dialogue}{Doc}
Plug documentation is not available.
\end{dialogue}
\begin{dialogue}{Man}
Headers, then.
\end{dialogue}
Two panes of plug documentation appear on one of the console monitors.
\begin{dialogue}{Man}
Ok, let's do this.
\end{dialogue}
\begin{dialogue}{Doc}
\end{dialogue}
\begin{dialogue}{Man}
\end{dialogue}
\begin{dialogue}{Doc}
\end{dialogue}
\begin{dialogue}{Man}
\end{dialogue}
\begin{dialogue}{Doc}
\end{dialogue}
\begin{dialogue}{Man}
\end{dialogue}

\theend
\end{document}
