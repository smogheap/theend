%%
%% This is file `example.tex',
%% generated with the docstrip utility.
%%
%% The original source files were:
%%
%% screenplay.dtx  (with options: `example')
%% +=+=+=+=+=+=+=+=+=+=+=+=+=+=+=+=+=+=+=+=+=+=+=+=+=+=+=+=+=+=+=+=+=+=+=+=+=
%% 
%% Authored by and Copyright (C)2006 by
%% John Pate <johnny@dvc.org.uk>
%% http://dvc.org.uk
%% Latest version is available as:
%% http://dvc.org.uk/sacrific.txt/screenplay.zip
%% 
%% This program is free software; you can redistribute it and/or
%% modify it under the terms of the GNU General Public License as
%% published by the Free Software Foundation; either version 2 of
%% the License, or (at your option) any later version.
%% 
%% This program is distributed in the hope that it will be useful,
%% but WITHOUT ANY WARRANTY; without even the implied warranty of
%% MERCHANTABILITY or FITNESS FOR A PARTICULAR PURPOSE.  See the
%% GNU General Public License for more details.
%% 
%% You should have received a copy of the GNU General Public
%% License along with this program; if not, write to the Free
%% Software Foundation, Inc., 51 Franklin St, Fifth Floor, Boston,
%% MA 02110-1301 USA
%% 
%% GPL v2 text also available as:
%% http://dvc.org.uk/gplv2.txt
%% 
%% +=+=+=+=+=+=+=+=+=+=+=+=+=+=+=+=+=+=+=+=+=+=+=+=+=+=+=+=+=+=+=+=+=+=+=+=+=
\documentclass{screenplay}[2006/11/15]

\title{Example .tex File}

\author{John Pate}

\address{the address\\
here if you're\\
going postal\\
UK\\
+44~(0)131~999~9999\\
johnny@dvc.org.uk
}

%% ... preamble finished, let's go ...
\begin{document}

%% Make a title page ...
\coverpage
%% ...or use \nicholl

%% TeX allows quite a lot of leeway in whitespace, so I've messed this
%% up a bit.  I find keeping the format structured helps me a lot tho.
%% Note: but don't have blank lines *inside* the body of text in
%% a dialogue environment.
%% I use vi (Elvis) with macros to make a lot of typing disappear.
%% You can, of course, define LaTeX macros to shorten the command names.
%%
%% Anyhoo, on with the show ...

%% for some reason this always happens at the start ...
\fadein

\intslug[illumination]{example sample -- screenplay.cls}

In space, nobody knows what time of day it is.  Wait, there is no
day.

So BOB, a cross-dressing Republican, and BROWN, a Christian
fundamentalist Democrat, are talking nonsense instead.

\begin{dialogue}{Bob}
That means that someone
sabotaged the unit and killed the
President! Was it one of us?
\end{dialogue}
\begin{dialogue}{Brown}
Who else is mad but us, Condi\ldots
 \paren{beat}
and Bliar?
\end{dialogue}

Bob buries his head in his hands.

\intslug{Atlantis -- somewhere ANyway}

JOHN and MARK are at adjacent consoles. FRED is with them. TOM
is at another console slightly further away.

\begin{dialogue}{John}
The planetoid seems to have a thin crust
covering a nickel-iron core. Could have
been an Earth-like planet at one time.
\end{dialogue}

\begin{dialogue}{Mark}
We're coming up on the radio source now.
\end{dialogue}

Brown walks in and goes to a console.

He has a PARROT on his shoulder.

The Parrot has an air of quiet insouciance.

\begin{dialogue}{Fred}
Switch the visual to main screen so we
can get a good look.\end{dialogue}

They look up at the main screen.

\begin{dialogue}[to John and Mark]{Fred}
Lock on to that.
\dialbreak[to Tom]{Fred}
Establish planetary orbit.
\end{dialogue}

\intextslug[day]{in or out}
Apparently some people do this.

\intercut
\extintslug[night]{out or in}
Or even this.

\extslug[day or nite]{NO WARRANTY -- EXPRESS OR IMPLIED}

\pov\ I made the slugline DAY/NIGHT optional 'cause in space no-one can
tell the time.  You probably will need to specify.

Don't put in pagebreaks by hand until you're really, really
finished editing!

It isn't the done thing to hyphenate for formatting purposes.
\centretitle{http://dvc.org.uk/sacrific.txt/}

That was a centred titleover.
\begin{titleover}There's a titleover environment for dialogue-like layout if you're
doing the "Star Wars" thing.\end{titleover}\extslug[All Hail Discordia!]{where to find us}
http://dvc.org.uk/sacrific.txt/screenplay.zip

Use the source, Luke.

\extslug[illumination]{for definitive info on layout}

http://www.oscars.org/nicholl/format.html

%% and this always happens at the end ...
\fadeout

\theend

\end{document}
%% 
%% Hail Eris!  All Hail Discordia!
%%
%% End of file `example.tex'.
